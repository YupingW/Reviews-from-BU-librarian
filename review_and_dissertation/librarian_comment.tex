
\documentclass[a4paper,11pt]{article}
\usepackage{hyperref}
\usepackage[inline]{enumitem}

\begin{document}
\noindent Greetings,






This is a lengthy message, I realize, but please read through all of it carefully, and to the very end, where you're instructed where and how to pay the dissertation fee.



I have extracted 27 pages from the file you sent to me and put yellow sticky notes on them where revisions are required.  Please double click on each note to open it fully.  Simply hovering the cursor over a note may not display all of the text within the note.



On pages xi and xii, you do not have to include all of the subsections for your C.V. in the Table of Contents.  You can save yourself a page by having just the entry for the C.V. as the last line of the contents.



Page xix, the list of abbreviations, there are many lines running over the right margin.  Use the \textbackslash sloppypar command to fix those lines.



Most of the main text is fine.  Tables 6.3, 6.4, and 6.5 on pages 128-130 are all very small and hard to read.  You should put each table on its own page in landscape (sideways) view so you can widen them to 8.5 inches and make them easier to read.  A table or figure is useless if you can�t read it.



On pages 143-145 you have alternating lines and tables that run over the right margin.  Use the \textbackslash sloppypar command for the long lines, and make the wide tables only six inches wide to fit within all required margins.



Most of the revisions are in your References section, where you were a bit too casual in referring to a number of conference proceedings.  A reference to ``In ICWSM'', ``In ACM IH'', and especially ``In WWW'' is obscure and won't mean anything to most people.  I spent three hours searching for each source and finding more complete information:  the full title, page numbers and the doi or URL, and pasted in each yellow note how I think those sources should be cited.  ACM will even provide you with a suggested citation on most of its pages.  You already have a URL in many of your references, so I fully expect you to be able to include the doi/URL I provide in each yellow note.



Some of the references you have by the first name of the main author instead of by his or her last name (e.g., the very first cite is for Lawrence Hurley instead of Hurley, L.  I don't even know why it's the first one, since Lawrence should have been with the L cites.  Please shift those around.  A few more lines of text run over the right margin and need to be fixed with the \textbackslash sloppypar command.  One author name is misspelled because I went to search for it and couldn't find the source with that name.  The cite for Chadwick I could not find.  I could find a book by that name from 2011, but nothing for ECPR.  I could find no reference online to the Common Crawl Repository.  I've never heard of it, and I have reviewed hundreds of ENG dissertations since 1998.



Something strange happened when I tried to highlight the article title for the Soroka cite.  Adobe seemed to put a red box around the text and then a blue comment bar appeared at the top of the window talking about ``unapplied redactions.''  No matter what I did, I couldn't get rid of it, but I don't think it matters because you will make revisions in your LaTeX file and save a new PDF that will not have that comment bar.



Once you have made all of those revisions and everything looks good to you, you can go to www.etdadmin.com/bu to create your ProQuest account, fill out the online paperwork, and upload the final version of the dissertation PDF.  If you save your own PDF, the system should tell you if there's a problem with it, mainly if the fonts aren't embedded. Please look at the PDF after you have created it, as odd things have been known to happen, such as blank, numbered pages.



I think I alerted you before about being sure to use only Type 1 fonts in your dissertation, and to embed all fonts.  If I did not warn you about those ProQuest requirements when I reviewed your approval page, tell me know.



When creating your account with ProQuest, please use your BU e-mail, but you're not required to use your Kerberos password.  You can create a new password just for ProQuest.  The online paperwork I believe is fairly straightforward.  If you do have questions, let me know.



We recommend that you include an ORCID iD when you submit your dissertation to ProQuest. The Open Researcher and Contributor ID (ORCID) registry provides persistent, non-proprietary identifiers that uniquely identify scientific and other academic authors and contributors. Graduate students and those who anticipate pursuing publication in their future careers should consider getting an ORCID now and using it often. Sign up here: https://orcid.org/register



Where it asks for your Institutional Student ID, that's your eight-digit BU ID, not your kerberos login name.  That's a field I cannot correct, so if you mess it up, I have to refer you back to the record to fix it.  Under the ProQuest Publishing Options and under IR (Institutional Repository) Publishing Options, you will see boxes to restrict the work for six months, a year, or two years.   If you want to restrict the work, in order to publish a journal article, for example, you must also submit a letter requesting that restriction, or embargo, signed by you and Professor Stringhini.  There�s more required of you than checking off a box.  If you do want to restrict the work, let me know and I will send you a file with the sample request letter.



I believe the whole part where they ask about the Creative Commons License relates to how much you want to allow other researchers to use and reproduce your work.  There's an explanatory page here https://creativecommons.org/licenses/  in case you want to investigate it further.  You can also do nothing for the Creative Commons License and leave it blank.  It�s up to you.



Where it asks if you want to upload your approval page and/or embargo or restriction letter, just ignore those.  I will upload that/those right before I deliver your dissertation to ProQuest.



Where it asks for the degree being awarded, make sure you select from the doctoral list and NOT the master's degree listings.  Once you do that, neither you nor I can go back and select a doctoral degree.  It's one of the clunkier aspects of a website I consider to be notorious for its clunkiness.  All I can do is wait until I deliver the PDF to ProQuest and then notify them to change the degree from M.A., or M.S., or whatever the confused student entered, to the correct doctoral degree.



In the Keywords section, please keep in mind that every word in your title and abstract is already searchable in the ProQuest database, so this section is only for words not in the title or abstract, if you can think of any.  If you cannot think of any new keywords, then leave the section blank and move on.  It�s intended to help someone find your dissertation more easily, not be a barrier to graduation.  When you get to the part where you�re asked to include your dissertation abstract, please just paste in the text body, and not the heading (title, author name, school name etc.).  The heading is extra and unnecessary at this point.



You can have ProQuest register copyright for you when you set up your online account with them, but registering copyright is not required.  The provost did decide in December, 2012 that all thesis and dissertation submissions should have the page, but registration of copyright is at the student's discretion. You should decide whether or not you want to register before setting up the account, as you get only the first chance when you go through the form to ask them to register copyright or not.  Ditto for if you want to order personal copies from them (there is a charge both for registering copyright and ordering personal copies, which you can put on a credit card).  If you realize later that you don�t want any copies, or want to order more than you did, you have to contact ProQuest customer service or you can withdraw the record and submit it all over again, which strikes me as tedious.



Please note: the page with the original signatures will NOT be included in the final dissertation submission.  It's a ProQuest and University policy to protect faculty from identity theft of having a scan of their signature available on the internet. A scan of the page is uploaded to the �Administrative Documents� section of your ProQuest record, but that is not for public view. Many students order personal copies from ProQuest when they upload their dissertation PDF, then pepper me with e-mails of complaint when those copies show up and they have only the unsigned version of the page in them.



I think that's everything.  When I do get the submission from you, I will assign it to Christine Ritzkowski in SE who will fill out her part of a checklist, and then re-assign it to George Zhang in ENG Graduate Records, who eventually assigns it back to me, as I said probably on commencement weekend in mid-May.  I take one last look at the PDF, and if it looks good, and you've paid the dissertation fee, I approve you for graduation.  You will receive e-mail notification of that.  I have your signed approval page.



Yes, the dissertation fee is \$115.  You can now go to http://www.bu.edu/library/guide/theses/ and about halfway down under the �Submit Electronically� subheading you should see a line to �Submit your Dissertation/Thesis processing fee online (requires BU Login)� (SEE third attached screen shot).  The University wants all graduate students to pay the fee that way. This fee is separate from any payments you may be charging with ProQuest.  Registering copyright or ordering personal copies are both optional fees.  The University dissertation fee is required.



If you have questions, let me know.



Best,

Brendan McDermott

Thesis/Dissertation Coordinator

Boston University Libraries

771 Commonwealth Ave.

Boston, MA  02215

%\bibliographystyle{plain}
%\bibliography{sigproc}

% Closing statement


\end{document}
%-----------------------------------------------------------------------------%
